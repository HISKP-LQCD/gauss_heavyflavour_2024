\rule{\textwidth}{0.4pt}

In many precision tests of the Standard Model (SM) of Particle Physics and
searches for New Physics beyond it, a precise understanding of hadron
structure and interactions is needed to provide ab initio predictions
to be compared with experimental data. The fundamental forces of Quantum
Chromodynamics (QCD) at low energy describe particles, namely mesons and
baryons, as the non-perturbative realisation of confined quarks interacting
via gluons. The only rigorous treatment of such non-perturbative
dynamics known to date is based on the regularisation of the theory on
a finite, discrete Euclidean space-time lattice, and subsequent
numerical simulations of the discretised theory. This allows one to
compute hadronic matrix elements of great phenomenological interest,
such as e.g those regulating hadron structure and fragmentation, decays and transition
amplitudes among different flavours of hadrons, and hadronic corrections
to electroweak precision observables.
%affecting precision observables, among others.

Flavour physics is currently one of the fields with a high potential
for the discovery of a statistically significant discrepancy between the SM and
experiment. Among others, there could be additional flavour-changing
interactions, further complex phases in the Cabibbo-Kobayashi-Maskawa (CKM)
matrix, or possible violations
of lepton-flavour universality. Even if the mass scales of new
particles beyond the SM turned out to be very high, quantum effects of
the associated fields could leave detectable imprints on the physics
of heavy quarks.
This makes precise theoretical computations of related observables
highly desirable. In particular, semi-leptonic decays of heavy mesons
and elements of the CKM matrix are very
interesting to estimate accurately from first principles.

Until recently, most lattice calculations focused on the investigation
of exclusive decays because inclusive processes consist
of a potentially very large number of physical states and their
systematic analysis in lattice QCD calculations was impractical.
However, novel approaches have been put forward that allow one to address
inclusive decays in lattice QCD, like the method described in
Ref.~\cite{Gambino:2020crt} where the total decay rate for the process
$D_s \to X\ell\bar\nu$ and $B \to X\ell\bar\nu$   is related to a smeared spectral
density that can be computed from an Euclidean four-point function in a finite
volume~\cite{Hansen:2019idp} (see
also \cite{Bulava:2019kbi,Bulava:2021fre,Gambino:2022dvu}). The computation of such
Euclidean correlation functions is standard and can be carried out
with high precision.
The reconstruction of the spectral densities
represents the main challenge that was already carried out for the $D_s$ meson
in the previous computer time allocation of this project and the results were
presented at the $41^{\mbox{st}}$ Lattice Conference at the University of Liverpool 
\cite{talklatt2024_ale, talklatt2024_chr}
and a paper is in preparation.

With this continuation of the computer time application, we propose to extend
the calculation of the total decay rate for the process $D_s \to X\ell\bar\nu$ to the
heavier $B$ and $B_s$ meson computing  $B (B_s) \to X\ell\bar\nu$.
As in the previous time contingent we are including the continuum limit
using the $N_f=2+1+1$ clover twisted-mass action with physical
values of up/down, strange and charm
quarks~\cite{ExtendedTwistedMass:2021qui,ExtendedTwistedMass:2021gbo,ExtendedTwistedMass:2022jpw}.
A necessary preliminary step to compute the inclusive decay rate of the $B_s$
meson is the determination of the 
$B$ physical point, i.e. the value of the bare parameter of the $b$-quark mass that reproduces the
experimental value of the $B$-meson mass.  
% quark bottom mass $m_b$.
This can be
achieved using the ratio method proposed in \cite{ETM:2009sed}
and successfully used in \cite{ETM:2016nbo}, where suitable ratios allow
to reach the bottom quark sector by combining simulations around the
charm-quark mass with an exactly known static limit.
The bare quark mass can then be converted in to renormilized values my multiply 
the appropriate renormalization constant $Z_p$ which computation by the ETMC 
is under finalization.


The continuation of the project to the $B_s$ meson is very interesting from the
experimental side  where recent results from $B$ factories reveal some tension with
SM predictions, but also exhibit puzzling discrepancies between exclusive and
inclusive channels~\cite{ParticleDataGroup:2020ssz, HFLAV:2019otj, Gambino:2019sif}
{\color{red} is there an update??}

%The theoretical study of semileptonic decays of heavy mesons encode direct information on the modulus of the elements of the
%Cabibbo-Kobayashi-Maskawa (CKM) quark mixing matrix \cite{Cabibbo:1963yz, Kobayashi:1973fv}. 
%By using the optical theorem, the total inclusive decay rate for the process $D_s \to X\ell\bar\nu$ can be written
%$|V_{cd}|$, $|V_{cs}|$ and $|V_{us}|$
%\begin{equation}
%	\Gamma = G^2_F\left\{ |V_{cd} |^2 \Gamma_{cd} + |V_{cs} |^2 \Gamma_{cs} + |V_{us} |^2 \Gamma_{su}
%	\right\}\,,
%\end{equation}
