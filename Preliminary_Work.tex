
The proposed project aims to continue GCS Large Scale projec ISDLQCD
(application number 28958) and the long term activities of the
Extended Twisted Mass Collaboration (ETMC) in the calculation of
hadronic matrix elements via lattice QCD simulations. ETMC has
produced and is still producing gauge configuration ensembles at
different lattice spacing values at the physical point, by which we
mean physical values of all the dynamical quark masses. For the
proposal here we propose to use four values of the lattice spacing,
corresponding to the ensembles compiled in Table~\ref{tab:ensembles}. The
ensembles labelled with $B$, $C$, $D$, $E$ have been produced
already. They are being used in several ETMC
works~\cite{ExtendedTwistedMass:2021qui,ExtendedTwistedMass:2021gbo,ExtendedTwistedMass:2022jpw,ExtendedTwistedMassCollaborationETMC:2022sta}
as well as in the previous period of this computer time project.
These four lattice spacing values will enable a controlled continuum
limit of the quantities relevant to this proposal.
Moreover, at the $B$ value of the lattice spacing we have
three volumes available, which will allow us to check finite volume
effects, which represent a main source of systematic uncertainty.
Since we work directly at physical light quark mass values, no
extrapolation in the light quark mass values is necessary.

For the continuum limit, we additionally benefit from automatic
$\mathcal{O}(a)$-improvement of the physical observables in question
thanks to the twisted mass action at
maximal twist, ensuring convergence to the continuum
limit with scaling violations of order $a^2$~\cite{Frezzotti:2003ni}.

\begin{SCtable}[.4]           
  \centering % center the table
  \begin{tabular}{lccccr} % alignment of each column data
    \toprule
    name          & $L$ [fm]      & $a$
    [fm]          & $M_\pi$ [MeV] & $M_\pi L$                         \\
    \midrule
    % cAp211.077.64 & 5.76 & $\sim$0.09 & $\approx$ 135 & 3.95  \\
    \midrule
    cB211.072.48  & 3.84          & $\sim$0.08 & $\approx$ 135 & 2.63 \\
    cB211.072.64  & 5.12          & $\sim$0.08 & $\approx$ 135 & 3.51 \\
    cB211.072.96  & 7.68          & $\sim$0.08 & $\approx$ 135 & 5.27 \\
    \hline
    cC211.06.80   & 5.44          & $\sim$0.07 & $\approx$ 135 & 3.72 \\
    % cC211.06.112  & 7.62          & $\sim$0.07 & $\approx$ 135 & 5.21 \\
    \hline
    cD211.054.96  & 5.76          & $\sim$0.06 & $\approx$ 135 & 3.94 \\
    \hline
    cE211.044.112 & 5.48          & $\sim$0.05 & $\approx$ 135 & 3.75 \\
    \bottomrule
  \end{tabular}
  \caption{ETMC's $N_f=2+1+1$ gauge ensembles relevant for this
    proposal. The time extent is always set to $T=2L$.}
  \label{tab:ensembles}
\end{SCtable}

The computation of the bottom quark mass has already been performed with
the previous generation of gauge ensembles of ETMC~\cite{ETM:2016nbo,ETM:2011zey} with
non physical pion masses, smaller volumes and coarser lattice
spacings.
The experience obtained then will be very useful in the proposed
project.
Moreover, that $B$ physics is possible on the ensemble compiled in
Table~\ref{tab:ensembles} was shown for instance in Ref.~\cite{Frezzotti:2024kqk}.

With the calculation of the $D_s\to X \ell \nu$ inclusive decay rate
in the current accounting period, which we are currently finalising,
we are certain that statistical and systematic uncertainties can be
controlled, while the $q^2$ dependence can be resolved at the same
time, also for $B$ and $B_s$ mesons.
Moreover, the computation of the decay rates at heavy quark masses in
the region between the charm and the bottom are interesting in itself:
it can be used to constrain the operator product expansion (OPE)
describing the inclusive semileptionic $B$
decay~\cite{Manohar:1993qn,Blok:1993va,Gambino:2004qm} following the 
technique of Ref.~\cite{Gambino:2022dvu}. 


\endinput
