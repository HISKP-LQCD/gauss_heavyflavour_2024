
The proposed project aims to continue GCS Large Scale projec ISDLQCD
(application number 28958) and the long term activities of the
Extended Twisted Mass Collaboration (ETMC) in the calculation of
hadronic matrix elements via lattice QCD simulations. ETMC has
produced and is still producing gauge configuration ensembles at
different lattice spacing values at the physical point, by which we
mean physical values of all the dynamical quark masses. For the
proposal here we propose to use four values of the lattice spacing,
corresponding to the ensembles compiled in Table~\ref{tab:ensembles}. The
ensembles labelled with $B$, $C$, $D$, $E$ have been produced already
and used in several ETMC
works~\cite{ExtendedTwistedMass:2021qui,ExtendedTwistedMass:2021gbo,ExtendedTwistedMass:2022jpw,ExtendedTwistedMassCollaborationETMC:2022sta}
an in the previous stage of this project.
These four lattice spacing values will enable a controlled continuum
limit of the quantities relevant to this proposal.
Moreover, at the $B$ value of the lattice spacing we have
three volumes available, which will allow us to check finite volume
effects representing a main source of systematic uncertainty.
Since we work directly at physical light quark mass values, no
extrapolation in the light quark mass values is necessary.

For the continuum limit we additionally benefit from automatic
$\mathcal{O}(a)$-improvement of the physical observables in question
thanks to the twisted mass action at
maximal twist, ensuring convergence to the continuum
limit with scaling violations of order $a^2$~\cite{Frezzotti:2003ni}.

\begin{SCtable}[.4]
	\centering % center the table
	\begin{tabular}{lccccr} % alignment of each column data
		\toprule
		name          & $L$ [fm]      & $a$
		[fm]          & $M_\pi$ [MeV] & $M_\pi L$                         \\
		\midrule
		% cAp211.077.64 & 5.76 & $\sim$0.09 & $\approx$ 135 & 3.95  \\
		\midrule
		cB211.072.48  & 3.84          & $\sim$0.08 & $\approx$ 135 & 2.63 \\
		cB211.072.64  & 5.12          & $\sim$0.08 & $\approx$ 135 & 3.51 \\
		cB211.072.96  & 7.68          & $\sim$0.08 & $\approx$ 135 & 5.27 \\
		\hline
		cC211.06.80   & 5.44          & $\sim$0.07 & $\approx$ 135 & 3.72 \\
		% cC211.06.112  & 7.62          & $\sim$0.07 & $\approx$ 135 & 5.21 \\
		\hline
		cD211.054.96  & 5.76          & $\sim$0.06 & $\approx$ 135 & 3.94 \\
		\hline
		cE211.044.112 & 5.48          & $\sim$0.05 & $\approx$ 135 & 3.75 \\
		\bottomrule
	\end{tabular}
	\caption{ETMC's $N_f=2+1+1$ gauge ensembles relevant for this
		proposal. The time extent is always set to $T=2L$.}
	\label{tab:ensembles}
\end{SCtable}

% In addition to the feasibility study presented by the authors of
% Ref.~\cite{Gambino:2022dvu} including some of the project contributors of this
% project, we have performed a first exploratory investigation of
% charm-down quark part of the
% differential decay rate of the $D_s$ meson decaying into $X\ell\bar\nu$ on
% the cB211.072.64 ensemble, for the first time at the physical
% point. The result is shown in Figure~\ref{fig:Gamma_B64}.

The computation of the bottom quark mass was already performed with
the previous generation of gauge ensembles of ETMC~\cite{ETM:2016nbo,ETM:2011zey} with
non physical pion masses, smaller volumes and coarser lattice spacing.
With the most recent ensembles of ETMC some $B$ physics has been studied, by some authors
that are present in this application, in \cite{Frezzotti:2024kqk}.
% but an hadronic scheme was used to determine the physical $B$ meson point, here we propose to 
% determine the renormalised bottom quark mass.   


The computation of the inclusive decay rate $D_s\to X \ell \nu$
is getting finilized and a preliminary result is reported in Figure~\ref{fig:dGammadq_Ds},
From  Figure~\ref{fig:dGammadq_Ds} we show that the statistical and systematic
uncertainty can be controlled  the
dependence on the squared momentum can be resolved. This makes us
confident that we can extend the calculation to heavier quark mass
using an extrapolation guided by heavy quark effective theory.
The computation of the decay rate at heavier masses will be interesting by itself even at lighter 
physical $B$ meson, because it can constrain the 
operator product expansion (OPE)  describing the inclusive semileptionic $B$ decay 
\cite{Manohar:1993qn, Blok:1993va, Gambino:2004qm} following the technique of \cite{Gambino:2022dvu}.

% To extend the calculation to the $B$ and $B_s$ meson we will simulate at six values of the 
% heavy quark in the range $1-3$ the charm quark mass and  extrapolate to the $b$ quark mass
%  to.


\endinput
