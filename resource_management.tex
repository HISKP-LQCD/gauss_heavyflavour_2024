\subsection{Resource management}

Our team has extensive experience in managing large computer time
allocations, having been awarded two PRACE Tier-0 allocations, many
NIC Gauss computing allocations, large scale production projects on
JUWELS Booster, Marconi100 at Cineca and Piz Daint at CSCS in
Switzerland, and a number of smaller local computer time
allocations. These allocations exceed 500 Mi core-hours in total. The
team will follow established practices for monitoring resources and
progress, including periodically (around once a week) telephone
conferences and local group meetings.  During these calls adherence to
the schedule proposed below will be evaluated, with corrections made
if necessary.

\subsection{Work schedule}

\begin{figure}[tbp]
  \begin{ganttchart}[
      x unit=0.335cm,        
      y unit title=0.5cm,
      y unit chart=0.5cm,
      vgrid,hgrid,
      %        title label anchor/.style={below=-1.6ex},
      bar label font=\small,
      title left shift=.05,
      title right shift=-.05,
      title height=1,
      incomplete/.style={fill=white},
      progress label text={},
      bar height=0.8,
      bar top shift=+0.1,
      group right shift=0,
      group top shift=.0,
      group height=.0
    ]{1}{48}
    %labels
    \sffamily
    \gantttitle{\textbf{Allocation duration}}{48} \\
    \gantttitle{Nov.}{4} 
    \gantttitle{Dec.}{4} 
    \gantttitle{Jan.}{4} 
    \gantttitle{Feb.}{4} 
    \gantttitle{Mar.}{4} 
    \gantttitle{Apr.}{4}
    \gantttitle{May.}{4} 
    \gantttitle{Jun.}{4} 
    \gantttitle{Jul.}{4} 
    \gantttitle{Aug.}{4} 
    \gantttitle{Sep.}{4} 
    \gantttitle{Oct.}{4}
    \\
\ganttbar[inline,bar/.style={fill=blue!30}]{cB64}{1}{8}\\
\ganttbar[inline,bar/.style={fill=blue!60}]{cB96}{8}{14}
\ganttbar[inline,bar/.style={fill=green!30}]{cC80 }{15}{24}
\ganttbar[inline,bar/.style={fill=magenta!30}]{cD96 }{24}{30}\\
\ganttbar[inline,bar/.style={fill=green!60}]{cC112 }{30}{40}
\ganttbar[inline,bar/.style={fill=red!30}]{cAp64 }{41}{48}
%    \gantttitle{Sub-project 1: pion and kaon $\langle x^n\rangle$}{48}\\
%    \ganttbar[inline,bar/.style={fill=red!30}]{cB64 connected 2- and 3-pt}{1}{48}\\
%    \ganttbar[inline,bar/.style={fill=blue!20}]{cB64 light/strange loops}{1}{48}\\
%    
%    \gantttitle{Sub-project 2: hadronic vacuum polarisation}{48}\\
%    \ganttbar[inline,bar/.style={fill=red!20}]{cC112 meson 2pt functions}{1}{48}\\
%    \ganttbar[inline,bar/.style={fill=blue!20}]{cB48 meson 2pt functions}{1}{16}
%    
  \end{ganttchart}
  \caption{A Gantt chart describing the scheduling of the steps
    involved in our calculation.}
  \label{fig:gantt}
\end{figure}

Since we are dealing in this proposal only with the estimation of
observables on already existing gauge configurations, all separate
runs are independent and, thus, runs for different configurations do
not depend on each other. We will avoid to store so-called propagators
completely, all contractions will be performed on the fly.

We plan to run all tasks sequentially as indicated in the Gantt
chart \Cref{fig:gantt}. 

\endinput
