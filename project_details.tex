\label{sec:proj}
The core of our project is the calculation of semi-leptonic decays of the $B_s$ mesons,
to do so we need to first determine the quark mass value $m_b$.
The raw data produced with the HPC facilities
contributes to the determination of
the total inclusive decay rate for the process $B_s \to X\ell\nu$
and the mass of the $B$ meson $M_B$. The simulations
will be done at several values of an heavy quark mass and the determination and
subsequent extrapolation to $m_b$ will be done in the analysis.

The computation proposed here will use the same ensembles of the previous stage of the project
i.e. Wilson-clover twisted
mass fermions~\cite{Alexandrou:2018egz}. The up/down quarks will be treated
in a fully unitary setting. In order to avoid flavour mixing lattice artifacts
in the unitary strange/charm sector of this regularisation, we use a
mixed action approach for strange and charm quarks: so-called
Osterwalder-Seiler type~\cite{Frezzotti:2004wz} strange and charm quark doublets
$(s^+ , s^-)^T$ and $(c^+ , c^- )^T$ are added with bare
twisted strange and charm quark mass $\pm \mu_s$ and $\pm \mu_c$
tuned to reproduce the physical mass of the $\phi$ and $J/\Psi$
mesons, respectively, as described in Appendix C of
Ref.\cite{ExtendedTwistedMass:2022jpw}. For more details, we refer to
this reference.

% Before discussing the two sub-projects we would like to point out that a
% successful implementation of the proposed project will produce not
% only phenomenologically interesting results, but it will also be the
% basis for a future extension towards B mesons using the ideas we have
% put forward in Ref.~\cite{ETM:2009sed}.

\subsection{Sub-project 1: Bottom quark mass $m_b$}
\label{sec:mb}

We can extract the mass of the $B$ pseudo-scalar ($M_{B_s}$) meson, with one bottom quark $q$
and a strange $s$,
from the right diagram of Figure~\ref{fig:4pt} representing the two point function
\begin{equation}
  \langle B_s(\bm{0},t) B_s^\dagger({\bf 0},0)\rangle\xrightarrow{t>>a, (T-t)>>a}
  \frac{1}{2M_B}|\langle 0 | B_s | B_s\rangle|^2\left(e^{-M_{B_s} t}+e^{-M_{B_s}(T-t)}\right)
  \, \label{eq:Mb}
\end{equation}
with the operator $B_s({\bf 0},t)=\frac{1}{L^3}\sum_{\bf x}\bar b\gamma_5 s({\bf x},t)$.
Due to present day computer limitations, it is not possible, however, to work directly
with the b-quarks propagating on the simulated lattice, the ratio method approach the
b-quark mass by interpolating from the charm region to the asymptotic infinite mass.
Thus, we compute Eq.~(\ref{eq:Mb}) replacing the b quark with several lighter quarks h
and for each we compute the quantity
\begin{equation}
  Q_m = \frac{M_{hs}}{M_{h\ell}^\gamma M_{cs}^{(1-\gamma)}}
\end{equation}
where $M_{hs}$ and $M_{hl}$ are the heavy-strange and heavy-light
pseudoscalar masses, respectively, while we denote by $M_{cs}$
the mass of the pseudoscalar meson made out of a charm
and a strange quark. The parameter $\gamma$ is a free parameter in the range  $[0, 1)$.
The asymptotic behavior can be computed using HQET
\begin{equation}
  \lim_{ m^{pole}_h\to \infty}
  \frac{M_{hs}}{( m^{pole}_h)^{(1-\gamma)} M_{h\ell}^\gamma}=\mbox{text}
  \label{eq:yHQFTlim}
\end{equation}
where $ m^{pole}_h$ is the pole mass of the heavy quark.
We then consider a sequence of heavy quark masses such that any two
successive masses have a common and fixed ratio i.e.
$ m_h^{(n)}=\lambda m_h^{(n-1)}, n=2,3,...$ and we construct the ratios
at given lattice spacing $a$
\begin{equation}
  y_Q( m^{(n)}_h,a)=\frac{Q_m( m_h^{(n)},a)}{Q_m( m_h^{(n-1)},a)}\cdot
  \left(\frac{ m_{h}^{(n)} \rho( m_{h}^{(n)})}{ m_{h}^{(n-1)}\rho( m_{h}^{(n-1)})}\right)^{(\gamma-1)}
  =\lambda^{(\gamma-1)}\frac{Q_m( m_h^{(n)},a)}{Q_m( m_h^{(n-1)},a)}\cdot
  \left(\frac{ \rho( m_{h}^{(n)})}{\rho( m_{h}^{(n-1)})}\right)^{(\gamma-1)}\,,
\end{equation}
where we have used the relation  $ m^{pole}_h= m_{h}^{} \rho( m_{h}^{})$
between the $\overline{\mbox{MS}}$ renormalized mass and the pole mass.
The factor $\rho$ are known perturbatively up to N$^3$LO \cite{Chetyrkin:1999pq}.
For each pair of heavy quark we carry out a continuum extrapolation
\footnote{We do not need a chiral extrapolation since we are already at physical point}
obtaining $y_Q( m^{(n)}_h)=y_Q( m^{(n)}_h,a=0)$.
In the continuum the ration $y_Q( m^{(n)}_h)$ can be described in therm
of heavy mass $ m_h$ as in \cite{ETM:2011zey}
\begin{equation}
  y_Q( m^{(n)}_h) = 1 + \frac{\eta_1}{ m_h}+ \frac{\eta_2}{ m_h^2}\,,
  \label{eq:fity}
\end{equation}
where the limit in Eq.~(\ref{eq:yHQFTlim}) is already incorporated.
Finally, the $b$-quark mass can be computed with
\begin{equation}
  y_Q( m^{(2)}_h)y_Q( m^{(3)}_h)...y_Q( m^{(K+1)}_h)\frac{Q_m( m_h^{(1)},a)}{\lambda^{K(\gamma-1)}}
  \left(\frac{ \rho( m_{h}^{(K+1)})}{\rho( m_{h}^{(1)})}\right)^{(1-\gamma)}=Q_m( m_h^{(K+1)},a)
  \,,
\end{equation}
where  ratios $y_Q$  in the lhs are evaluated using the fit function Eq.~(\ref{eq:fity})
and the parameters $\lambda$, $K$ and $ m_h^1$ are such that $Q_m( m_h^{(K+1)},a)$ matches
the experimental value of $M_{B_s}/(M_{B}^\gamma M_{D_s}^{(1-\gamma)})$.



%%%%%%%%%%%%%%%%%%%%%%%%%%%%%%%%%%%%%%%%%%%%%%%%%%%%%%%%%%%%%%%%%%%%%%%%%%%%%%%%%%
%%%%%%%%%%%%%%%%%%%%%%%%%%%%%%%%%%%%%%%%%%%%%%%%%%%%%%%%%%%%%%%%%%%%%%%%%%%%%%%%%%
\subsection{Sub-project 2: total inclusive decay rate $B_s \to X\ell\nu$}

By using the optical theorem, the total inclusive decay rate for the
process $B_s \to X\ell\nu$ can be written as
\begin{equation}
  \Gamma = G^2_F\left\{ |V_{bu} |^2 \Gamma_{bu} + |V_{bc} |^2 \Gamma_{bc} + |V_{bu} |^2 \Gamma_{bu}
  \right\}\,,
\end{equation}
where the different contributions to the right side correspond at the quark level
to the weak transitions $b \to u$, $b \to c$ and $b \to u$ respectively. Each contribution can be written as
\begin{equation}\label{eq:Gamma_fg}
  \Gamma_{fg}=\int \frac{d^3p_\nu}{(2\pi)^32E_\nu}\frac{d^3p_\ell}{(2\pi)^32E_\ell}
  L_{\mu\nu}(p_\ell, p_\nu) H^{\mu\nu}_{fg}(p,p-p_\ell-p_\nu)\,,
\end{equation}
where the leptonic tensor is given by
\begin{equation}
  L_{\mu\nu}(p_\ell, p_\nu) =4\left\{p_\ell^\mu p_\nu^\nu +p_\ell^\nu
  p_\nu^\mu - g^{\mu\nu} p_\ell\cdot p_\nu+
  i\epsilon_{\mu\nu\alpha\beta} p_\ell^\alpha p_\nu^\beta\right\}\,,
\end{equation}
while the hadronic tensor reads
\begin{equation}
  H^{\mu\nu}_{fg}(p,p_X)=\frac{1}{2m_{D_s}}\langle D_s| J^\mu_{fg}(0)(2\pi)^4
  \delta^4(\mathbb{P}-p_x) J^{\nu\dagger}_{gf} (0)| D_s\rangle\,.
\end{equation}
$\mathbb{P}$ is the QCD four momentum operator and $J_{gf}$ are the
relevant Minkowski weak currents $J_{gf}^\mu(x)=i\bar
  g(x)\gamma^\mu(1-\gamma_5)f(x)$.
On the contrary of the decay of the $D_s$ meson here there are not disconnected
diagrams. The contribution $\Gamma_{bu}$ is suppressed by the integration over
the phase space of \eqref{eq:Gamma_fg} and it is expected to not contribute to the total
decay rate.

From now on we will drop the indices $fg$ and the procedure we are describing will be equivalent for the $bu$, $bc$ and $bu$ contribution,
moreover in the rest frame of the $B_s$ meson we have $p=(M_{B_s},\bm{0})$ thus we simplify the notation of
$H^{\mu\nu}(p_X)\equiv H^{\mu\nu}_{fg}(p,p_X)$.
The hadronic tensor $H_{\mu\nu}$ is
the spectral density of the quantity
\begin{equation}
  M_{\mu\nu}(t,{\bf q})= \int_0^{\infty}d \omega H_{\mu\nu} (\omega,{\bf q}) e^{-\omega t}\,,\quad \text{with}\quad p_X=(\omega,\bm{q})\,.
\end{equation}
$H_{\mu\nu}$ can be reconstructed from $M_{\mu\nu}$ using the method presented in
Ref.~\cite{Hansen:2019idp}. $M_{\mu\nu}$ can be directly computed from the
ratio of Euclidean lattice correlators
\begin{gather}
  M_{\mu\nu}(t_2-t_1,{\bf q})=\frac{C_{\mu\nu}(t_{snk},t_2,t_1,t_{src};q)}{e^{-(t_{snk}-t_2)}  C(t_1-t_{src}) }\label{eq:ratio_4pt_2pt}\\
  \label{eq:4pt}
  C_{\mu\nu}(t_{snk},t_2,t_1,t_{src};q)=\int d^3x e^{i{\bf q}\cdot {\bf x}}
  \langle D_s({\bf 0}, t_{snk}) J^{\dagger}_\mu({\bf x},t_2)  J_\nu(0,t_1)
  D_s^\dagger({\bf 0}, t_{src})\rangle\,,\\
  \label{eq:2pt}
  C(t_{snk}-t_{src}) =  \langle D_s({\bf 0}, t_{snk})
  D_s^\dagger({\bf 0}, t_{src})\rangle\,.
\end{gather}
In general, the inverse problem represented by the extraction of
hadronic spectral densities from Euclidean correlators is notoriously
ill-posed. Recently, a method to cope with these problems has been
proposed in Ref.~\cite{Hansen:2019idp}. It consists of treating the
integrals mentioned above with some $C_\infty$ kernel $K_\sigma$
\begin{equation}
  H^\sigma_{\mu\nu}(\omega^*, {\bf q})= \int d \omega K_\sigma( \omega^*,\omega ) H_{\mu\nu}(\omega, {\bf q})\,,
\end{equation}
making the inverse problem well posed and $H^\sigma_{\mu\nu}$ computable.
In our case the phase space integration in \eqref{eq:Gamma_fg}
provides a sharp smearing kernel function $\theta$. Following
Refs.~\cite{Gambino:2020crt, Gambino:2022dvu}, the phase space integral over
\eqref{eq:Gamma_fg} can be written as
\begin{gather}
  \frac{48 \pi^4}{m_{D_s}^5}\frac{d\Gamma}{d \bm{ \omega^2} }
  =\sum_{l=0}^2 |\bm{\omega}|^{3-l}\int_0^{\infty}d \omega_0 \Theta^l(\omega_0^{max}-\omega_0) Z^l\,,\quad\quad {\omega}_0^{max}=1-|\bm{\omega}|\,,\quad\Theta^l(x)=x^l\theta(x)\\
  Z^2=Y_3-2Y_1\,,\quad Z^1=2(Y_3-2Y_1-Y_4)\,,\quad Z^0=Y_2+Y_3-2Y_4\,.
\end{gather}
Moreover, the $Y_i$ can be directly computed from the Euclidean
hadronic tensor $H^{\mu\nu}$ as follows
\begin{align}
                                                     & Y^1=-m_{D_s}\sum_{ij}\hat{n}^i\hat{n}^j H^{ij}(p,q)\,,          &  & Y^2=-m_{D_s}H^{00}(p,q)\,, \\
                                                     & Y^3=m_{D_s}\sum_{ij}\hat{\omega}^i\hat{\omega}^j H^{ij}(p,q)\,, &  &
  Y^4=-im_{D_s}\sum_{i}\hat{\omega}^i H^{0i}(p,q)\,, &
\end{align}
with $\hat{n}$ a unit vector orthogonal to $\bm\omega$.
Thus the phase space integral provides us a $\theta$-function smearing
kernel. However, the $\theta$-function is not smooth, which is why we
will use a $C_\infty$ kernel that
after taking the infinite volume limit, the
$\theta$-function is recovered as $\sigma\to0$, i.e. $\lim_{\sigma\to 0} K_\sigma^l(x^*,x)=\Theta^l(x^*-x)$.
%such that the step-function is recovered in the limit $\sigma\to0$, i.e.
%$\lim_{\sigma\to 0} \theta_\sigma(x)=\theta(x)$.
Therefore,
\begin{gather}
  \frac{48 \pi^4}{m_{D_s}^5}\frac{d\Gamma}{d \bm{ \omega^2} }
  =\lim_{\sigma\to 0}\sum_{l=0}^2 |\bm{\omega}|^{3-l}\int_0^{\infty}d \omega_0 K_\sigma^l(\omega_0^{max},\omega_0) Z^l\,.
\end{gather}
The spectral reconstruction can be performed at the analysis stage. Thus,
the main cost of this calculation is the production of the
four-point function \eqref{eq:4pt}. In the case of the $b\to u$ weak
transition the corresponding required Wick contractions are shown in
Fig.~\ref{fig:4pt}.

The renormalization constants necessary to renormalise the ratio of
correlators \eqref{eq:ratio_4pt_2pt} $Z_A$ and $Z_V$ are already
computed in previous work of ETMC~\cite{ExtendedTwistedMass:2022jpw}.

\begin{figure}
  \centering
  \begin{tikzpicture}
    \begin{feynman}[scale=1.5]
      \vertex[anchor=east,blob, fill=black!0!] (a) at (0,0) {\(B_s^\dagger\)};
      \vertex[anchor=east,blob, fill=black!0!]  (J1) at (1,0.8) {\(J_W\)};
      \vertex[anchor=east,blob, fill=black!0!] (J2) at (2.4,0.8) {\(J_W\)};
      \vertex[anchor=west,blob, fill=black!0!] (b) at (3,0) {\(B_s\)};
      \vertex (s1) at (3.4,-0.4) ;
      \vertex (s2) at (3.4, 0.4) ;
      \vertex (s3) at (2.6, 0.7) ;
      \vertex (s4) at (1.8, 1) ;

      \diagram*{
      %				(a) --[fermion,  half right,looseness=0.5,edge label=$s$] (b);
      (b) --[fermion, red, out=220, in=320,edge label=$s$,swap] (a);

      (a) --[fermion, green!40!black, edge label=$b$] (J1) ;
      (J1) --[fermion, blue, edge label=$u$, reversed momentum'=$\bf{q}$] (J2);
      (J2) --[fermion, green!40!black ,edge label=$b$] (b);
      (s1) --[->, half right, edge label=sequential,swap] (s2);
      (s3) --[->, half right, edge label=sequential,swap] (s4);
      };
    \end{feynman}
    \begin{feynman}[scale=1.5]
      \vertex[anchor=east,blob, fill=black!0!] (a) at (6.5,0) {\(B_s^\dagger\)};
      \vertex[anchor=west,blob, fill=black!0!] (b) at (9.5,0) {\(B_s\)};

      \diagram*{
      %				(a) --[fermion,  half right,looseness=0.5,edge label=$s$] (b);
      (b) --[fermion, red, out=220, in=320,edge label=$s$,swap] (a);
      (a) --[fermion, green!40!black, out=40, in=140,edge label=$b$,swap] (b);
      };
    \end{feynman}
  \end{tikzpicture}
  \caption{On the left: Wick contractions required for the calculation of the correlator \eqref{eq:4pt} for the contribution $cd$. For the contribution $cs$ the $d$-quark propagator must be replaced with and $s$-quark propagator. On the right: Wick contractions required for the calculation of the correlator \eqref{eq:2pt} .}
  \label{fig:4pt}
\end{figure}

A similar analysis was done in the first stage of the project and the result
are shown in Fig.~\ref{fig:dGammadq_Ds} already extrapolated to the continuum and
to $\sigma\to0$.
The extra difficulties will come from the fact we can not simulate directly at the
physical $b$-quark, but the decay rate will be computed for a series of masses lighter the $b$-quark and the result extrapolated to the physical $b$ mass computed in the sub-project 1 of section~\ref{sec:mb}.

% As mentioned already, we have performed a first exploratory
% investigation of the differential decay rate on ensemble cB211.072.64
% in order to understand the feasibility of the proposed calculation. We
% used 300 gauge configurations with three stochastic sources each
% (see below for how stochastic sources are used here)
% and we compute only the $cd$ contribution. The
% resulting differential decay rate is shown in Fig.~\ref{fig:dGammadq_Ds}
% for four values of the square momentum of the final hadronic state. Apart from the statistical
% accuracy, which appears satisfactory, we also wanted to understand how
% to choose the time-separations appropriately. For one of the values of
% the momenta we have performed the calculation for different choices
% for the times $t_{snk}$, $t_2$ and the minimal value of $t_1$ included
% in the spectral reconstruction. However, within errors, all three
% choices give compatible results.

% For the calculation proposed here we plan to stick to the most
% conservative choice, namely $t_{snk}=56$, $t_2=40$ and $t_1=16$,
% corresponding to the blue circles in Fig.~\ref{fig:dGammadq_Ds}.
% For the ensembles at the other lattice spacings and volumes we plan to keep the
% time separations constants in physical units.


\begin{figure}
  \centering
  \includegraphics[scale=0.7]{plots/dgamma_dq2.png}
  \caption{Preliminary values of the differential decay rate of
    $D_s\to X\ell\nu$ (only the contribution coming from $\Gamma_{cd}$) as a function of the squared momentum for the
    extrapolated to the continuum limit..}
  \label{fig:dGammadq_Ds}
\end{figure}

% \subsection{Sub-project 2: decay constant $f_D$ and $f_{D_s}$}

% From the two-point function \eqref{eq:2pt} we can extract the decay
% constant of the corresponding pseudo-scalar (PS) meson as 
% \begin{equation}
%   f_\mathrm{PS}=(\mu_f+\mu_{f'})\frac{\langle 0| P_{ff'}|
%     \mathrm{PS}\rangle}{M_\mathrm{PS}^{ff'}\sinh(M_\mathrm{PS}^{ff'})}\, 
% \end{equation}
% where the matrix element is extracted from the correlation function
% \eqref{eq:2pt} at large time separations. The PS-meson is made out of
% valence quark flavours with bare masses $\mu_f$ and $\mu_{f'}$ , and
% its mass is denoted by $M_\mathrm{PS}^{ff'}$. We plan to compute the
% decay constant for the $D_s$ and $D$ meson, thus $f=c$ and $f'=s,d$. 

\endinput
